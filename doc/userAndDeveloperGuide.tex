\begin{filecontents}{helpTable.tex}
\begin{longtable}{l|>{\raggedright\hsize=0.3\hsize}X|>{\hsize=1.7\hsize}X<{\vspace{4pt}}}
%\begin{longtable}{l|l|X<{\vspace{4pt}}}
    \caption{Mabs commands.} \\
    Command & Arguments & Description \\
    \hline \endhead
    \mcommand{?}, \mcommand{h}, \mcommand{help} & --- & Outputs help to
    stdout \\
    \mcommand{about} & --- & Outputs the 'about' information to
    stdout \\
    \mcommand{addMatrix},\mcommand{am} & \mparam{Files} & Include PSSMs from
    files matching \mparam{Files}. \\
    \mcommand{addSequence},\mcommand{as} & \mparam{Files} & Include all
    sequences from FASTA formated files matching \mparam{Files}. \\
    \mcommand{addSingleSequence}, \mcommand{ass} & \mparam{File} & Include the
    only sequence in gzipped and FASTA formatted \mparam{File}. This might be
    useful for very large sequences.\\
    \mcommand{align} &
    \mparam{File}, \mparam{N}, \mparam{$\lambda$}, \mparam{$\xi$}, \mparam{$\mu$}, \mparam{$\nu$}, \mparam{nPR} &
    Executes the pairwise alignment procedure. All of the parameters
    are optional and value '.' gives the parameter its default
    value. See section~\ref{sec:pairalign} for details. \\
    \mcommand{more} & \mparam{N} & Fetch \mparam{N} more alignments
    from the previous alignment. \\
    \mcommand{getTFBS} & \mparam{Cutoff} & Search for transcription factor
    binding sites with PSSMs from the included DNA sequences. The
    scoring \mparam{Cutoff} is relative to the maximum score of each
    PSSM. See sec.~\ref{sec:TFBSscan} for details.\\
    \mcommand{getTFBSabsolute} & \mparam{Cutoff} & Search for transcription factor
    binding sites with PSSMs from the included DNA sequences. The
    scoring \mparam{Cutoff} is for absolute score independent of PSSM
    or sequence. See sec.~\ref{sec:TFBSscan} for details.\\
    \mcommand{printMatrices}, \mcommand{pm} & --- & Outputs the PSSM frequencies
    to stdout. \\
    \mcommand{printMatrixWeights}, \mcommand{pmw} & --- & Outputs the
    PSSM weights to stdout. \\
    \mcommand{printSeqNames}, \mcommand{ps} & --- & Outputs the
    names of the included sequences to stdout. \\
    \mcommand{quit}, \mcommand{q} & --- & Quit mabs. \\
    \mcommand{removeMatrix}, \mcommand{rm} & \mparam{k} & Removes
    matrix number \mparam{k}. \\
    \mcommand{removeSequence}, \mcommand{rs} & \mparam{name} & Removes
    the sequence called \mparam{name}. \\
    \mcommand{reset} & --- & Removes all matrices and sequences. \\
    \mcommand{resetMatrices},\mcommand{resm} & --- & Removes all matrices. \\
    \mcommand{resetSequences},\mcommand{ress} & --- & Removes all sequences. \\
    \mcommand{setMarkovBG} & \mparam{bgSample}, \mparam{order} & If
    \mparam{bgSample} is a sequence name, create new
    \mparam{order}-order Markov background
    using that sequence as data. If \mparam{bgSample} is a previously
    stored Markov background datafile, use that.\\
    \mcommand{setBGfreq} & \mparam{A}, \mparam{C}, \mparam{G}, \mparam{T}
    & Set background nucleotide frequencies. This is default and this
    removes the Markov background. \\
    \mcommand{setpseudocount} & \mparam{C} & Sets the amount of
    pseudo counts added to PSSMs when counting the weights. \\
    \mcommand{showmatch}, \mcommand{sm} & --- & Outputs the results
    from TFBS scan to stdout. \\
    \mcommand{savematch} & \mparam{File} & Saves the results
    from TFBS scan to \mparam{File}. \\
    \mcommand{savealign} & \mparam{File} & Saves the fetched alignment
    results to \mparam{File} in fancy human readable format. \\
    \mcommand{savealignAnchor} & \mparam{File} & Saves the fetched alignment
    results to \mparam{File} in anchor format understood by
    DIALIGN2\cite{Morgenstern99}. \\
    \mcommand{savealignGFF} & \mparam{File} & Saves the fetched alignment
    results to \mparam{File} in simple machine readable format. \\
    \mcommand{showalign}, \mcommand{sa} & --- & Outputs the fetched
    alignment results in fancy human readable format to stdout. \\
    \label{tab:commands}
  \end{longtable}
\end{filecontents}







\documentclass[12pt,a4paper]{article}

\title{Users and Developers Guide for MABS program}
\author{Kimmo Palin}
\date{\today}

\newcommand{\Prob}{\mathrm{P}}
\newcommand{\DNAalpha}{\ensuremath{\Sigma}}
\newcommand{\BSalpha}{\ensuremath{\Gamma}}
\newcommand{\Score}[1]{\ensuremath{s({#1})}}
\newcommand{\sPos}[1]{\ensuremath{p({#1})}}
\newcommand{\ePos}[1]{\ensuremath{q({#1})}}
\newcommand{\sdist}[2]{\ensuremath{\delta(#1,#2)}}
\newcommand{\bsType}[1]{\ensuremath{c({#1})}}

\newcommand{\mcommand}[1]{\emph{\sf #1}}
\newcommand{\mparam}[1]{'\emph{#1}'}

%\usepackage{tabularx}
\usepackage{ltxtable}
\usepackage{lscape}

\begin{document}
\maketitle


\section{Introduction}
\label{sec:intro}

Mabs is a comparative genomics tool for detecting enhancer
modules conserved in related species. The design goal for mabs has
been its applicability for detecting mammalian enhancers, especially
enhancers for human genes. The main feature posed by mammalian
genome is that they have very long contigs with high percentage of
``junk'' DNA.

Mabs differs from traditional local alignment tools by concentrating
its attention only to the functional sites on the DNA. These sites are
detected with position specific scoring matrices\cite{Stormo00} (PSSMs) that are provided by
the user. These PSSMs usually represent the binding affinity of
a~transcription factor to a~particular DNA sequence. 
The other data the user has to provide is obviously the paralogous DNA
sequences themselves.

\section{Usage}
\label{sec:usage}

Mabs has evolved to somewhat complex piece of software. The program is
controlled by written commands with additional information provided by
parameters. The commands can be given either on command line (prepended
with '-') or on interactive command shell. The most up-to-date help
about all available commands can be obtained with
command~\mcommand{help}. The complete list of mabs commands along with
brief description of each can be found in table~\ref{tab:commands}.

The commands from the command line are read from left to right, such
that the command%
\begin{verbatim}!# mabs -as *.fa -am TFv1/*.pfm -setMarkovBG human.ChrI.O4.bg \\
        -getTFBSabsolute -savematch \end{verbatim}%
searches all '*.fa' files (\mcommand{\mbox{-as} *.fa}) for binding sites from
directory 'TFv1' (\mcommand{\mbox{-am} TFv1/*.pfm}) with respect to a~Markov
background (\mcommand{\mbox{-setMarkovBG} human.ChrI.04.bg}). All sites scoring
better than the default cutoff of 9, are searched
(\mcommand{\mbox{-getTFBSabsolute}}) and stored to a~file with a~name similar to
 'mabs\_2003\_8\_27\_15\_48.gff' (\mcommand{\mbox{-savematch}}).



\section{Finding Potential Binding Sites}
\label{sec:TFBSscan}

The \mcommand{getTFBS} and \mcommand{getTFBSabsolute} commands look
for the sites whose nucleotide distribution differ the most from the
background sequence toward the given position specific
frequencies. The user inputs the position specific counts $C[i,c]$,
$1\le i \le m$ and $c\in\{A,C,G,T\}=\DNAalpha$ in
the PSSM files given with the \mcommand{addMatrix} command. The PSSM
files should contain four rows of non negative integers, each number
representing the count of the given nucleotide in that position of the
site. Each row stands for one nucleotide so that the first row has
counts for 'A', the second row has counts for 'C', the third row for
'G' and the fourth row for 'T'.  The sum of each column does not need
to be the same.


Each character of~$\DNAalpha$ can occur in each position of the
background sequence with probability~$p_A$, $p_C$, $p_G$, $p_T$ (which
can be set with \mcommand{setBGfreq}).  The position specific
nucleotide distribution of the binding site motif ($M$)is biased toward the
background with pseudo count~$s$ (set by \mcommand{setpseudocount})
which can be seen as a~Dirichlet prior (see
e.g.~\cite{durbin98}). This also alleviates some technical difficulties
with division by zero later on. The effect of the pseudo count
diminishes when there are more observations on the correct motif,
i.e. the counts $C[i,c]$ are larger. The probability, or the weighted
frequency, of the binding site (Motif) position~$i$, $M_i$ having a
character $c$ is
\begin{equation}
  \label{eq:motifProb}
  \Prob(M_i=c)=\frac{C[i,c]+p_cs}{\sum_{x\in\DNAalpha}C[i,x]+s}
\end{equation}

The binding sites are searched with respect to two optional background
distributions. The simpler one assumes the background DNA as
independently and identically distributed sequence of characters. This
means that the probability of character~$c$ in background sequence
position~$i$, $B_i$, is
\begin{equation}
  \label{eq:bg0markov}
  \Prob(B_i=c)=p_c
\end{equation}.

The more complicated background distribution takes in to account the
positional dependence between the nearby nucleotides in the background
sequence. This dependence is noted with~$k$:th order Markov model,
where~$k$ is the number of nucleotides considered when estimating the
probability of character $B_i$.  This probability
$\Prob(B_i|B_{i-1},\ldots,B_{i-k})$ is computed on-line during the
search by using precomputed data (set by \mcommand{setMarkovBG}). The
Mabs distribution includes a file 'human.ChrI.04.bg' which
contains~4th order Markov background model learned from human
chromosome one.

For both types of background distribution the motif matching score is
computed the same way. The score is the log likelihood ratio between
the motif and the background. For sequence $S_l,\ldots,S_{l+m}$ the
score is
\begin{equation}
  \label{eq:bsScore}
  \begin{array}{lll}
  \log_2
  \frac{\Prob(S_l,\ldots,S_{l+m}|M)}{\Prob(S_l,\ldots,S_{l+m}|B)} &=& \log_2
  \frac{\prod_{i=1}^{m}\Prob(M_i=S_{l+i-1})}{\prod_{i=1}^m
    \Prob(B_i=S_{l+i-1}|S_{l+i-2},\ldots,S_{l+i-k}) }\\
   &= &\sum_{i=1}^{m}\log_2
  \frac{\Prob(M_i=S_{l+i-1})}{
    \Prob(B_i=S_{l+i-1}|S_{l+i-2},\ldots,S_{l+i-k}) }
  \end{array}
\end{equation}

For the iid background distribution, the score is simply a sum of
weights since it only depends on the position on the motif and the
character in the string.
\begin{equation}
  \label{eq:bg0markovScore}
\sum_{i=1}^m \log_2
  \frac{\Prob(M_i=S_{l+i-1})}{\Prob(B_i=S_{l+i-1})}.
\end{equation}
These weights can be precomputed so that the actual binding site scan
over the sequence is only summing over a moving window.

The higher order Markov background is a bit trickier since the weights
depend on previous characters in the string. But still we can use the
moving sum method for the probability of the motif and compute the
background probability independently. After computing both of the log
likelihoods, we can get the score by subtracting
\begin{equation}
  \label{eq:markovCounting}
  \sum_{i=1}^{m}\log_2
  \Prob(M_i=S_{l+i-1}) -  \sum_{i=1}^{m}\log_2
    \Prob(B_i=S_{l+i-1}|S_{l+i-2},\ldots,S_{l+i-k}).
\end{equation}
In the beginning of the sequence the background probability is
computed with lower order Markov chain.

\section{Pairwise alignment}
\label{sec:pairalign}

The novel feature in mabs tool is the local binding site alignment. It
shares ideas with the global binding site
alignment of Blanco~et.al.~\cite{Blanco_etal03} and with the local
sequence alignment of Smith and Waterman~\cite{SmithWaterman81}. our
goal is to detect the evolutionary conserved gene enhancer elements
from genomic DNA sequence.

The \mcommand{align} command takes seven optional arguments, all of
which can be omitted with character '.' (period) on its place. First
parameter \mparam{File} is the GFF file \mparam{File} is the GFF file containing the binding site
information produced with the
\mcommand{savematch} command containing the binding site
information. The file should contain at binding sites for at least two
sequences. If more than two sequences are given in one file, the
program asks two sequences that should be aligned (This can not be
given directly with commands, parameters nor from command line). If
the \mparam{File} parameter is omitted, the binding site data is
obtained from the TFBS scan preceding the \mcommand{align} command.

The second parameter~\mparam{N} gives the number of best local
alignments to return after the alignment run. More suboptimal
alignments can be requested with \mcommand{more} command.

The last four parameters \mparam{$\lambda$}, \mparam{$\xi$},
\mparam{$\mu$}, \mparam{$\nu$} and \mparam{nPR} control the alignments
weighting function. The mabs local alignment attempts to estimate the
energy provided and required by transcription factor proteins binding
to DNA sequence. Some of the binding energy is transformed to work
when the DNA is twisted and bent to correct conformation allowing
proper binding.

Lets consider the input for the alignment function as two strings $A=a_1
\ldots a_n$ and~$B$ %= b_1 \ldots b_n$ 
of binding sites represented as tuples $(\bsType{a_i}, \Score{a_i}, \sPos{a_i},
\ePos{a_i})$. The $\bsType{a_i}$ is a character from the
alphabet~$\BSalpha$ that represents all
known binding site motifs.  Each individual tuple contains also the score
$\Score{a_i}$ and the start and end positions $\sPos{a_i}$ and
$\ePos{a_i}$ on the DNA sequence respectively. 
%All of this data is
%available in the GFF files produced with \mcommand{savematch} command
%and required in aligning two binding site sequences $A$ and $B$.


The score $\Score{a_i}$, obtained most likely with \mcommand{getTFBS}
or \mcommand{getTFBSabsolute} commands, is our estimate for binding
affinity of the transcription factor to binding site $a_i$. The
estimated binding energy
is obtained by multiplying the score by the parameter~$\lambda$. While
aligning sites~$i$ and~$j$ from two sequences~$A$ and~$B$, we have
\begin{equation}
  \label{eq:scoreReward}
  \lambda(\Score{a_i}+\Score{b_j}).
\end{equation}

To estimate the energy consumed in the binding, we first assume that
the synergy between two adjacent binding sites are linearly dependent
on their distance from each other. To simplify the notation, when
aligning $a_k$ and $a_i$ next to each other, we denote their distance
\begin{equation}
  \label{eq:deltaNotation}
  \sdist{a_k}{a_i}=\sPos{a_i}-\ePos{a_k}.
\end{equation}
Now, when aligning $a_k$ and $a_i$ with $b_l$ and $b_j$, we subtract
from the binding energy the estimate for ``synergy'' work
\begin{equation}
  \label{eq:linearPenalty}
  \mu\frac{\sdist{a_k}{a_i}+\sdist{b_l}{b_j}}{2}.
\end{equation}

Since the binding sites in one particular enhancer region work
synergisticly it would be conceivable that the distances between the
binding sites within the enhancer should stay relatively the same. The
difference in the distance between the two adjacent binding sites
results in two different energy demanding tasks. For one, the binding
sites have to be bent back to their optimal distance in space possibly
forming a loop in the DNA. We estimate the energy required by this by
the energy required to bend an elastic rod. This energy is dependent
on the square of the distance bent and inverse of the distance over
which the bending is done. In total, we estimate the energy by
\begin{equation}
  \label{eq:bendingPenalty}
  \nu \frac{(\sdist{a_k}{a_i}-\sdist{b_l}{b_j})^2}{\sdist{a_k}{a_i}+\sdist{b_l}{b_j}}
\end{equation}

The second energy drain resulting from the altered distance is the
helical structure of DNA~\cite{WatsonCrick53}. Since the protein binds
to the major groove of the DNA, the protein binding site must be in
correct~3D orientation if it simultaneously interacts with other
proteins also bound to the same DNA. We estimate the energy required
by the twisting of the DNA also with the model of elastic rod, and get
a formula
\begin{equation}
  \label{eq:twistingPenalty}
  \xi \phi^2/(\sdist{a_k}{a_i}+\sdist{b_l}{b_j})  
\end{equation}
where $\phi$ is the twisting angle in radians, assuming the DNA has
\mparam{nPR} nucleotides per complete rotation.

Our alignment method tries to find a common subsequence from two binding site
sequences $A$ and $B$ such that the subsequence has maximal binding
energy, computed by summing~(\ref{eq:scoreReward}) and
subtracting~(\ref{eq:linearPenalty}), (\ref{eq:bendingPenalty})
and~(\ref{eq:twistingPenalty}) for each aligned binding site pair
$a_i$, $b_j$. This subsequence can be found by dynamic programming
with using recurrence formula
\begin{equation}
  \label{eq:recurrence}
  \begin{array}{ll}
D_{i,j} = \max_{ 0\le k <i, 0\le l<j} \left\{ 0,\right. & %
    D_{k,l}+\lambda(\Score{a_i}+\Score{b_j}) -  \\%
    &  \mu\frac{\sdist{a_k}{a_i}+\sdist{b_l}{b_j}}{2} - %
    \nu
    \frac{(\sdist{a_k}{a_i}-\sdist{b_l}{b_j})^2}{\sdist{a_k}{a_i}+\sdist{b_l}{b_j}} - \\
    
    & \left.  \xi \phi^2/(\sdist{a_k}{a_i}+\sdist{b_l}{b_j})\right\}.
  \end{array}
\end{equation}
for $i$ and $j$ such that $\bsType{a_i}=\bsType{b_j}$. If $a_i$ are $b_j$ sites
for different transcription factor, the $D_{i,j}$ is infinite and can
thus be disregarded in the alignment computation.
The maximum energy subsequence is found by backtracking from the matrix
position obtaining the highest value.

The brute force implementation of recursion
formula~(\ref{eq:recurrence}) would result an algorithm with time
complexity~$O(n^4)$ and space complexity~$O(n^2)$ where~$n$ is the
average length of the binding site sequences. These quite high
complexities can be avoided by careful implementation. Since we are
only interested on $D_{i,j}$ such that $a_i$ and $b_j$ are for the
same transcription factor, we can implement~$D$ as a sparse matrix
storing only matching positions. This decreases the expected space complexity
to $O(n^2/|\BSalpha|^2)$ if we assume uniform distribution of binding
sites. The worst case complexity is not affected.

The time complexity can be decreased by restricting the attention to
adjacent sites with limited distance between them. While this
restriction does alter the scoring function~(\ref{eq:recurrence}) it
is biologically plausible to assume that binding sites very far apart
in the genome do not interact. This biological assumption limits the
work done for each filled cell of~$D$ by a~constant thus bringing the
time complexity down to optimum with respect to the space
requirement. In expectation that is $O(n^2/|\BSalpha|^2)$.

The significance of the mabs alignment scores is somewhat a
guestion. We recomend to search for reasonable alignment score cutoff
values by pairwise aligning non related sequences, for example two
sequences from the same organism. The best local alignment of this
kind of ``dummy'' alignment procedure can be used as lower bound of
alignment score significance.

\section{Acknowledgments}
\label{sec:acknow}

The problem and the weighting function is provided by Jussi Taipale
along with the biological background of mabs methods.

We thank Matthias Berg for coding the initial version of the mabs
program and laying foundations for mabses user interface and core
design.

\bibliographystyle{apalike}
\bibliography{all}


\appendix

\begin{landscape}
\section{Commands}
\label{sec:mabscommands}
\begin{small}
\LTXtable{\linewidth}{helpTable.tex}%
\end{small}
\end{landscape}


\section{Version history}
\begin{verbatim}
$Log$
Revision 1.2  2004/01/29 09:14:45  kpalin
Binding site scanning section added.

Revision 1.1  2004/01/28 08:43:38  kpalin
Initial introduction and command help.

\end{verbatim}
\end{document}
